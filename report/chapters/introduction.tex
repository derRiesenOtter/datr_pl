\chapter{Introduction}
Heat stress poses a significant challenge to organisms by damaging cellular 
structures and functions. In response, cells activate the heat shock response, 
an ancient and conserved mechanism involving the production of heat shock proteins. 
These molecular chaperones prevent protein aggregation, assist in protein refolding, 
and restore cellular homeostasis. Triggered by protein unfolding rather than direct 
temperature sensing, this response mitigates widespread cellular disruptions, 
including cytoskeletal damage, organelle disorganization, and ATP depletion. 
\cite{richter_heat_2010}

This work focuses on the analyses of gene expression data provided by a yeast 
heat shock experiment that concentrated 
on the binding protein Mip6. The researchers 
that conducted this experiment tested the heat shock response on a single 
culture of \textit{Saccharomyces cerevisiaes} that was split into three groups. 
The control group, that was maintained at $30^\circ\text{C}$ and two groups 
that were incubated at $39^\circ\text{C}$ for 20 (Condition 1) and 120 (Condition 2)
minutes. The RNA-seq data gathered in this experiment will be analysed here. 
\cite{nuno-cabanes_multi-omics_2020}

In our analyses we will cover the genetic response of yeast to heat shock by 
determining which classes of genes were up- and down-regulated. We will also 
invest if there are clusters of genes that show a similar response and examine 
their general function using the R package WGCNA. We will also compare the results 
of the tools Kallisto and Bowtie2 used to create the alignments as well as the 
results for differential gene expression created by EdgeR and Limma.

