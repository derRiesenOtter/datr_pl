\chapter{Discussion}

\section{Genetic Response to Heat Shock in Yeast}

The yeast \textit{Saccharomyces cerevisiae} responds to heat shock 
through significant changes in gene expression. Heat shock predominantly 
upregulated genes involved in protein folding and stress response, 
such as chaperones or components crucial for ribosomal activity, 
while downregulated genes were often associated 
with retrotransposons and certain dubious open reading frames. Up regulation 
generally occured more often and stronger.

\section{Clusters of Genes and Their Functions}

Weighted Gene \gls{wgcna} identified 
one significant gene module correlating with heat shock conditions. 
Using a word cloud it was possible to get a overview of functions 
of the genes within this cluster. Important keywords such as transcription, 
synthase or polymerase hint at a strong relation of this cluster with 
the creation of new proteins.

\section{Comparison of Tools and Agreement of Results}

Gene expression quantification using Kallisto and Bowtie2 yielded 
comparable TPM distributions, with both methods indicating higher 
expression levels under heat shock conditions. However, discrepancies 
arose in identifying significantly altered genes, as evidenced by Jaccard 
similarity coefficients (0.60 and 0.40 for conditions 1 and 2, 
respectively). Differential expression analysis also highlighted 
differences: Limma (with voom normalization) exhibited a broader 
detection of high log-fold changes compared to EdgeR, which identified 
fewer but more symmetric log-fold changes. The absent overlap in identified 
upregulated genes but moderate agreement for downregulated genes 
(Jaccard coefficients of 0.48 and 0.18 for conditions 1 and 2) underscores 
the variability introduced by different analytical pipelines. 
