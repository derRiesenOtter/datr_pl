\chapter{Methods}
The analysis was conducted on the BioServer of the TH Bingen running 
Ubuntu 22.04.2 (x86) as well as on a private laptop running MacOS 15.1.1 (arm). 
\section{Programs}
The programs and versions listet in table \ref{tab:programs} were used for the analysis.

\begin{table}[H]
	\center
	\caption{Programs used for the analysis}\label{tab:programs}
	\begin{tabular}{ccl}
		\toprule

    Program       & Version & Operating System        \\
		\midrule
    bash          & 5.1.16 & Ubuntu       \\
		bowtie2       & 2.4.4   & Ubuntu      \\
		fastqc        & 0.11.9    & Ubuntu    \\
		Stringtie & 2.2.1      & Ubuntu   \\
		gffread       & 0.12.8      & Ubuntu  \\
		kallisto      & 0.46.2       & Ubuntu \\
		R             & 4.4.0        & Ubuntu \\
		Samtools      & 1.13         & Ubuntu \\
		slurm-wlm          & 21.08.5 & Ubuntu \\
		Trimmomatic   & 0.39         & Ubuntu \\
    bash          & 5.2.37    & MacOS    \\
    R & 4.4.2 & MacOS \\
    ggVennDiagram             & 1.5.2  & MacOS       \\
    ggplot2           &   3.5.1   & MacOS   \\
    wordcloud            &   2.6  & MacOS    \\
    WGCNA           &   1.73   & MacOS   \\
    tximport          &     1.34.0 & MacOS   \\
    stringr        &     1.5.1  & MacOS  \\
    edgeR           &   4.4.1    & MacOS  \\
    RColorBrewer         &  1.1-3   & MacOS    \\
    limma         &  3.62.2    & MacOS   \\
    tidyverse         &  2.0.0    & MacOS   \\
    corrplot         &  0.95    & MacOS   \\
    rtracklayer         &  1.66.0 & MacOS \\
		\bottomrule
	\end{tabular}
\end{table}

\section{Reference Transcriptome}
To create a reference transcriptome we used gffread on our reference genome. 
We calculated the number of generated reference
transcriptomes, checked if all transcripts in the gff file have been translated 
to reference transcripts in the output file
and analysed the number of splice variants for each gene.

\section{Quality Filtering}
To ensure that only reads with a sufficient quality were used for our analysis we used Trimmomatic to remove
adapter sequences, low-quality bases, and other contaminants. We calculated the number of genes filtered out
and compared the quality of the reads before and after this step with FastQC.

\section{Gene Expression Quantification}
We quantified the gene expression using Kallisto and Bowtie2.
Samtools was used to create a binary file from the result file of bowtie2. Stringtie was used
to generate the gene counts for our bowtie2 results.
For Kallisto we calculated the total number of expressed genes as well as the percentage of expressed
genes relative to the total in the transcriptome. Furthermore, we also analysed some basic statistics and
calculated the Correlation between the samples.
For bowtie2 we also calculated the percentage of aligned reads as well as analysed the
alignment quality statistics.
For both methods we extracted the \gls{tpm} values on a per-gene basis and compared them. 
We then calculated the log fold change for the \gls{tpm} values and also compared them between the 
methods.

\section{WGCNA}
Using the results of the Kallisto gene expression quantification we performed a \gls{wgcna} to construct a gene network and identify modules of high 
correlation. We also identified gene modules showing significant correlations with
heat-shock conditions and created a Wordcloud using the Gene Ontology Term Annotations. 


\section{Differential Gene Expression}
For the differential gene expression analysis we also used the results of Kallisto.
We conducted this analysis with edgeR as well as Limma (with the voom normalisation). 
We identified increased and decreased genes in these differentially expressed gene 
and again created word clouds. 

