\documentclass[11pt, a4paper]{scrreprt} % fontsize, format, art des dokuments

\usepackage[utf8]{inputenc} % ermoeglicht Verwendung von Unicode
\usepackage{graphicx} % zum einfuegen von grafiken
  \graphicspath{ {figures/} } % gibt an wo die Grafiken sind
\usepackage[german]{babel} % zum umstellen auf deutsch
\usepackage{csquotes} % wegen babel, zum zitieren mit "" sonst warnung
\newcounter{savepage} % Zähler zum weiterzählen der römischen Zahlen
\usepackage[hidelinks]{hyperref}
\usepackage{float}
\usepackage{booktabs}
\usepackage{tabularx} % zum leichteren nutzen von tabellen
\setlength{\parindent}{0em} % indent wegmachen
\usepackage{listings}
\usepackage[linesnumbered,ruled,vlined]{algorithm2e}
\usepackage{algorithmic}
\usepackage{amsmath}
\usepackage{float}
\usepackage[acronym]{glossaries}
\usepackage{subcaption} % praktisch, wenn man eine Grafik aus zwei Grafiken oder mehr macht

%Setting counters for citations, tables and figures
\AtBeginDocument{\counterwithin{lstlisting}{section}%
             \renewcommand\thelstlisting{\arabic{lstlisting}}}
\usepackage[style=numeric, backend=biber, sorting = none]{biblatex} % gute Moeglichkeit seine Quellen zu verwalten
  \addbibresource{bibliography/resources.bib} % In dieser Datei liegen dann die Quellen
\counterwithout{figure}{chapter} % Figures werden absolut gezaehlt und nicht nach Kapitel
\counterwithout{table}{chapter} % Tables werden absolut gezaehlt und nicht nach Kapitel



\makeglossaries

\newacronym{COVID 19}{COVID 19}{coronavirus disease 2019}

\newacronym{SARS-CoV-2}{SARS-CoV-2}{severe acute respiratory syndrome coronavirus type 2}

\newacronym{plDDT}{plDDT}{predicted local distance difference test}

\newacronym{CASP}{CASP}{Critical Assessment of Protein Structure Prediction}

\newacronym{GPU}{GPU}{graphics processing unit}

\newacronym{AI}{AI}{artificial intelligence} 

\newacronym{RMSD}{RMSD}{root-mean-square deviation} 

\newacronym{Cameo3D}{Cameo3D}{Continuous Automated Model EvaluatiOn} 

\newacronym{Phyre2}{Phyre2}{Protein Homology/analogY Recognition Engine V 2.0} 

\newacronym{RNP}{RNP}{ribonucleocapsid}

\newacronym{RNA}{RNA}{ribonucleic acid}

\newacronym{DNA}{DNA}{deoxyribonucleic acid} 

\newacronym{NSP}{NSP}{Non-structural protein} 

\newacronym{ACE2}{ACE2}{angiotensin-converting enzyme 2} 

\newacronym{BFD}{BFD}{Big Fantastic Database} 

\newacronym{MSA}{MSA}{Multiple Sequence Alignment} 

\newacronym{HMM}{HMM}{Hidden Markov Model} 

\newacronym{ENA}{ENA}{European Nucleotide Archive} 

\newacronym{PDB}{PDB}{Protein Data Bank} 

\newacronym{PSI-BLAST}{PSI-BLAST}{Position-Specific Iterative Basic Local Alignment Search Tool}

\newacronym{VRAM}{VRAM}{video random access memory} 

\newacronym{RAM}{RAM}{random access memory} 

\newacronym{HDD}{HDD}{hard disk drive} 

\newacronym{NVMe}{NVMe}{nonvolatile memory express} 

\newacronym{SSD}{SSD}{solid state drive} 

\newacronym{lDDT-C}{lDDT-C}{local distance difference test-C}

\newacronym{mmCIF}{mmCIF}{macro-molecular Crystallographic Information File} 

\newacronym{PNG}{PNG}{portable network graphic} 

\newacronym{URL}{URL}{uniform resource locator} 

\newacronym{iCn3D}{iCn3D}{I see in 3D} 


\begin{document}

	\pagenumbering{Roman}
	\begin{titlepage}

	\begin{figure}
	\includegraphics[width=1\textwidth]{th_bingen.png}
	\end{figure}
	
	\begin{center}
	\large
	TH Bingen Technical University of Applied Sciences\\
    Department 2 - Technology, Informatics and Economics\\
    Applied Bioinformatics (B.Sc.)\\
    \vspace{0.5cm}
	
    \huge
    \textbf{Gene Expression Analyses on yeast heat
shock experiments} 
    \vfill
    
    \large
    Prüfungsleistung DATR Mining with R \\
    Abgegeben am: ???\\
    Name: Robin Ender
    Matrikelnummer: 2184737
    \vspace{0.5cm}
	\end{center}
	
\end{titlepage}


	The Abstract is a short summary of the scientific publication that usually
contains a short description of the publications topic to help the reader to
decide whether to read the full paper or not. Optimally the reader is convinced
to read the full paper after the author generated enough interest. The Abstract
prepares the readers for the detailed an in-depth processes of the topic. At
last, an abstract helps the reader to remember key points from the publication.
Citations must be done in this style:

	\listoffigures
	\listoftables
	\tableofcontents
	\cleardoublepage
	
	\setcounter{savepage}{\arabic{page}} 
	\pagenumbering{arabic}
	\chapter{Introduction}

\section{questions to answer}
The Introduction mostly answers the following questions in a short manner:
\begin{itemize}
	\item What is the topic of this scientific publication?
	\item What is the current point of research?
	\item What materials and which methods were used to work on the topic?
	\item What is the hypothesis, that was to confirm or deny?
    \item This is an example for a citation (\cite{Beispiel1}). % Siehe dazu auch ./bibliography/resources.bib

\end{itemize}

\section{Introduction style}
An Introduction is written in plain text, and is usually written as one
paragraph of continuing text, although it is also possible to subdivide the
text into different partitions.

\subsection{This is a sub-section}

Lorem ipsum bla foo bar...

  \chapter{Material}
\section{List of topics}
\label{Topics_of_Material}
Multilevel List of the Data and Substances content:
\begin{enumerate}
   \item What kind of data has been used and in which version?
   \begin{itemize}
     \item Database XY
     \begin{description}
	 \item[XY's function and description:] the database was used in the version
			 2002002.2.456 and only contains the latest premium curated data,
			 just like SILVA f.e. 
     \item [Genomes:] the used genomes were the genes C137-A, C137-B and C137-C
     \end{description}
   \end{itemize}
   \item Which substances and organisms have been used? 
   \begin{itemize}
     \item Phage in genetically modified version 
     \begin{description}
	 \item [Function:] the phages function was to carry a plasmid ring with
			 modified rna into the victim organism 
	 \item [Variants:] Version 1442.312.1.666
     \end{description}
   \end{itemize}
\end{enumerate}

	\chapter{Methods}
\section{Reproduction}
Throughout the whole paper there is a need to describe every detail and every
step in a precise manner to make sure all created results can be reproduced by
others. First and foremost this reproduction demand has to be ensured in the
"methods" chapter.

Also very important are:
\begin{itemize}
	\item A chronological order of all steps (including software versions,
	details, data-handling, ..) 
	\item What method was used to solve each part of the task?  
	\item What data versions created which results?
\end{itemize}

\begin{figure}[ht]
    \centering
    \includegraphics[width=1\textwidth]{figures/alphafold_architecture.png}
    \caption{Example of a captiones figure}
    \label{modelarch}
\end{figure}
This is a single-line code example:
\begin{verbatim}
    $ docker run --rm --gpus all nvidia/cuda:11.0-base nvidia-smi
\end{verbatim}
\section{Formula examples}
Scores below 0.2 belong to randomly selected unrelated proteins, according 
to stringent statistics of structures in the \acrshort{PDB}, while those with
more than 0.5 generally have the same protein fold (\cite{Beispiel1}) oder \parencite{Beispiel1}. The TM-score
is calculated defined by:
\[ TM-score = max 
\begin{bmatrix}
\label{complex_formula_1}
\frac{1}{L_{target}}\sum_{i}^{L_{common}} \frac{1}{1+(\frac{d_i}{d_0(L_{target})})^2}
\end{bmatrix}\]
Where L$_{target}$ is the length of the amino acid sequence of the target
protein, and L$_{common}$ is the number of residues that appear in both the
template and target structures, d$_{i}$ is the distance between the $i$th pair
of residues in the template and target structures, and $d_{0} (L_{target})$ is
a distance scale that normalizes distances and is defined by:
\[ d_{0} (L_{target}) = 1.24\sqrt[3]{L_{target}-15-1.8} \]

This is a refererence to the labeled (numbered) formula [\ref{complex_formula_1}]
When comparing two protein structures that have the same residue order, L$_{common}$ 
reads from the C-alpha order number of the structure files. The \acrshort{RMSD}
value is defined by: \[ RMSD = \sqrt{\frac{1}{N}\sum_{i=1}^{N} \delta i^2} \]
Where $\delta$i is the distance between atom i and either a reference structure 
or the mean position of the N equivalent atoms.


\begin{table}[H]
  \begin{center}
    \caption[Distribution of values tested during simulated annealing in 4th
    quartile of high scoring parameter sets]{Distribution of values tested during
    simulated annealing in 4th quartile of high scoring parameter sets (mean
    F2-Scores $>$ 0.6454).}
    \begin{tabular}{c r r r r r r r}%
      \toprule{}%
      Parameter & Minimum & 1st Quartile & Median & Mean \\
      \midrule{}%
      $\alpha$        & 0.1000 & 0.2000 & 0.4000 & 0.4692     \\
      $\beta$         & 0.0476 & 0.3704 & 0.4545 & 0.4622     \\
      $\omega$        & 0.0476 & 0.1250 & 0.2105 & 0.2324     \\
      $\sigma$        & 0.0476 & 0.1875 & 0.3077 & 0.3054     \\
      Sprot-$w$       & 10.0000 & 20.0000 & 30.0000 & 41.02009738 \\
      Sprot-$\delta$  & 0.1000 & 0.3000 & 0.5000 & 0.5365    \\
      trEMBL-$w$      & 10.0000 & 50.0000 & 70.0000 & 67.10005704 \\
      trEMBL-$\delta$ & 0.1000 & 0.5000 & 0.7000 & 0.6823     \\
      TAIR-$w$        & 10.0000 & 30.0000 & 50.0000 & 53.79005853 \\
      TAIR-$\delta$   & 0.1000 & 0.3000 & 0.5000 & 0.5392     \\
      \bottomrule{}%
    \end{tabular}
    \label{tbl:value_table}
  \end{center}
\end{table}

This is an example on how to reference a table that has the
\verb|\label{tbl:value_table}| by using \verb|\ref{tbl:value_table}|: The
estimated values are listed in table \ref{tbl:value_table}.

	\chapter{Results}

\section{Reference Transcriptome}

From the 6585 genes of \textit{Saccharomyces cerevisiae}, 11599 transcripts have been generated. 
Of these 6585 genes, 2575 had one splice variant and 4512 had two.

\section{Quality Filtering}

Only a small number of reads were filtered out (up to $\sim 1$ percent).
The quality improvement was not visible in the data provided by 
FastQC.

\section{Gene Expression Quantification}
\subsection{Kallisto}
For each replicate around ten thousand (mean of 10622) of the 11599 trancripts have 
been found. This corresponds to over 90 percent. The \gls{tpm} values between the samples were comparable
between the samples. The control condition had slightly lower median values (between 9 and 12) compared to 
condition 1 (between 15 and 16) and condition 2 (between 14 and 18). 
They ranged between 0 and 86654.40. To visualize them the $log(TPM+1)$ was used, see
Figure~\ref{fig:box_kallisto}.

\begin{figure}[H]
  \center
  \includegraphics[width=0.8\textwidth]{6_3_kallisto_boxplot.pdf}
  \caption{Box plots of the TPM values for the three replicates ot the Kallisto analysis.}\label{fig:box_kallisto}
\end{figure}

The correlation between the technical and biological replicates of the Kallisto 
alignment was calculated and visualized in 
Figure~\ref{fig:corr_kallisto}.
The samples of condition 1 showed a high correlation between each other. The replicates for 
the control condition as well as for condition 2 differed more in comparison and did not 
form clusters. 

\begin{figure}[H]
  \center
  \includegraphics[width=0.7\textwidth]{6_3_kallisto_corr_matrix.pdf}
  \caption{Correlation plot for the technical and biological replicates of the Kallisto
  analysis. Clusters are shown by black borders.}\label{fig:corr_kallisto}
\end{figure}

\subsection{Bowtie2}

For the Bowtie2 alignment we analysed how many reads were mapped once, multiple times or were 
not aligned at all. All replicates showed similar ratios. 
The mapping of the reads of Bowtie2 is shown in Figure~\ref{fig:bar_bowtie}

\begin{figure}[H]
  \center
  \includegraphics[width=0.7\textwidth]{6_3_bowtie_alignment_bar.pdf}
  \caption{Mapping of the reads of the Bowtie2 alignment.}\label{fig:bar_bowtie}
\end{figure}

\subsection{Comparison Kallisto and Bowtie2}
The distribution of the \gls{tpm} values obtained by Kallisto and Bowtie2 were similar. 
The Bowtie2 results also showed a slightly lower median for the control group.
For both programs and for both conditions the log-fold changes have been calculated 
and the distribution was plotted in Figure~\ref{fig:dist_lfc}.
The distribution of gls{tpm} values looks comparable between the programs. A shift towards positive 
log-fold changes can be seen. 

\begin{figure}[H]
  \center
  \includegraphics[width=0.7\textwidth]{6_post_all_lfc_density.pdf}
  \caption{Grouped box plots of the three biological replicates per condition and per program.}\label{fig:dist_lfc}
\end{figure}

To see if Kallisto and Bowtie2 identified the same genes as significantly changed, 
we calculated the Jaccard similarity coefficient for genes with an absolute log-fold 
change~$> 2$ for condition 1 (0.60) and for condition 2 (0.40). 

\pagebreak

\section{WGCNA}
During the \gls{wgcna} one significant module eigenegene
was identified. To see which genes are found in this module a word cloud using GO annotations 
was created, see 
Figure~\ref{fig:worcloud_wgcna}.

\begin{figure}[H]
  \center
  \includegraphics[width=0.7\textwidth]{10_6_wordcloud_me2.pdf}
  \caption{Word Cloud for the Module Eigengen 2.}\label{fig:worcloud_wgcna}
\end{figure}

\section{Differential Gene Expression}

\subsection{Comparing edgeR and Limma}

Venn diagrams were created showing the relation of the log-fold change 
and the adjusted p-values. As seen in Figure~\ref{fig:venn}, 
edgeR and Limma behaved differently. While in edgeRs results only a few genes show a 
log-fold change greater than 2, almost all genes in Limmas results do. The Limma 
genes also seem to have lower adjusted p-values. 

\begin{figure}[H]
    \centering
    \begin{subfigure}[b]{0.45\textwidth} 
        \centering
        \includegraphics[width=\textwidth]{11_edgeR_venn_diagram_cn1.pdf} 
        \caption{}
        \label{fig:venn_edgeR_1}
    \end{subfigure}
    \hfill % Horizontal space between subfigures
    % Second subfigure
    \begin{subfigure}[b]{0.45\textwidth} % Adjust width as needed
        \centering
        \includegraphics[width=\textwidth]{11_limma_venn_diagram_cn1.pdf} % Replace with your image
        \caption{}
        \label{fig:venn_edgeR_2}
    \end{subfigure}
    \caption{Venn diagrams displaying relations between the adjusted p-value (FDR) and the log-fold 
    change (LFC) for the edgeR results (a) and Limma results (b) for condition 1.}
    \label{fig:venn}
\end{figure}

Viewing the volcano plots edgeR shows a relatively symmetric log-fold change, with mostly 
low values. The volcano plot for the Limma results on the other hand shows a clear tendency 
towards positive and higher log-fold changes.

\begin{figure}[H]
    \centering
    \begin{subfigure}[b]{0.45\textwidth} 
        \centering
        \includegraphics[width=\textwidth]{11_edgeR_volcano_con1.pdf} 
        \caption{}
        \label{fig:vol_edgeR_1}
    \end{subfigure}
    \hfill % Horizontal space between subfigures
    % Second subfigure
    \begin{subfigure}[b]{0.45\textwidth} % Adjust width as needed
        \centering
        \includegraphics[width=\textwidth]{11_limma_volcano_con1.pdf} % Replace with your image
        \caption{}
        \label{fig:vol_edgeR_2}
    \end{subfigure}
    \caption{Volcano plots visualizing the log-fold change 
    and significance for each gene for edgeR (a) and Limma (b) for condition 1.}
    \label{fig:vol_edgeR}
\end{figure}

Jaccard similarity between edgeR and limma was calculated for the top twenty up and down regulated genes. No overlap 
was found for the up regulated genes. However, this was not the case for the down regulated genes.
For condition 1 a Jaccard similarity coefficient of 0.48 was calculated, for condition 2 a Jaccard 
similarity coefficient of 0.18 was calculated.

\subsection{Differentially expressed genes}

For the top twenty up and down regulated genes word clouds were generated. 
Both edgeR and limma show an up regulation of proteins associated with heat shock, 
limma also shows an up regulation of chaperone associated proteins. 
For the down regulated genes the words 3-isopropylmalate and dehydratase were often found.

Using the descriptions provided by the \textit{Saccharomyces} Genome Database \cite{noauthor_saccharomyces_nodate}
we looked at the most up and down regulated genes. We found up regulation for genes related to 
plasma membrane organization during stress conditions (YFL014W), as well as genes that 
take part in the production of proteins like tRNAs (YNCB0013W) or rRNA(YNCL0012C, YNCL0021C).
For down regulated genes we found genes associated with retrotransposons (YHR214C-C, YDR365W-A) 
and several dubious open reading frames (YER152W-A, YGL152C, YGL239C).




	\chapter{Discussion}
\begin{itemize}
	\item What is the meaning of the results? (interpretation included here)
	\item How robust is the experiment and what are the weak links?
	\item How reliable are the results?
	\item What are possible next steps? $\Rightarrow$ New hypotheses /
			perspectives / follow-up experiments
\end{itemize}

	\cleardoublepage
	
	\pagenumbering{Roman}
	\setcounter{page}{\thesavepage}
	\printbibliography
	
	\cleardoublepage
	\pagenumbering{gobble}
	\leavevmode
\vfill

\begin{center}
\textbf{Erklärung zur Originalität der Arbeit}
\end{center}
Hiermit bestätige ich, dass die abgegebene Arbeit das Original ist und von mir ohne weitere Hilfe geschrieben wurde. Wenn Arbeit anderer referenziert oder genutzt wurde, wurde dies angemessen kenntlich gemacht. Meine Arbeit wurde noch nicht bewertet oder veröffentlicht. Die elektronisch abgegebene Version stimmt mit der elektronischen überein.

\vspace{1cm}
\hfill
\begin{tabular}[t]{c}
  \rule{10em}{0.4pt}\\ Unterschrift
\end{tabular}
\hfill
\begin{tabular}[t]{c}
  \rule{10em}{0.4pt}\\ Ort und Datum
\end{tabular}
\hfill
\strut
\vspace{2cm}


\begin{center}
\textbf{Erklärung zum Eigentum und Urheberrecht}
\end{center}
Hiermit erkläre ich meine Zustimmung, dass die Technische Hochschule Bingen diede Arbeit anderen Studierenden und interessierten Dritten zur Verfügung stellen und in meinem Namen (Robin Ender) veröffentlichen darf.

\vspace{1cm}
\hfill
\begin{tabular}[t]{c}
  \rule{10em}{0.4pt}\\ Unterschrift
\end{tabular}
\hfill
\begin{tabular}[t]{c}
  \rule{10em}{0.4pt}\\ Ort und Datum
\end{tabular}
\hfill
\strut
\vfill


	
\end{document}
